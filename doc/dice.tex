\documentclass{article}

% Input packages & formatting
% Packages

% Math packages
\usepackage{amsmath} % Extended math functions
\usepackage{amssymb} % Extended math symbols (loads in amsfonts)
\usepackage{bm} % Bold math symbols
\usepackage{mathtools}

% Figure packages
\usepackage{caption} % Caption formatting for university standard
\usepackage{graphicx} % includegraphics command
\usepackage{subcaption} % Subfigures
\usepackage[section]{placeins} % Place floats in section
\usepackage{wrapfig}

% Table packages
\usepackage{booktabs} % Better tables
\usepackage{bigstrut} % Merged table cells
\usepackage{longtable} % Tables which overflow into next page
\usepackage{array}
\usepackage{colortbl} % Color table cells
\usepackage{makecell}
\usepackage{multirow}

% Fonts
\usepackage{lmodern} % Use latin modern rather than computer modern. Better for font encoding.
\usepackage[T1]{fontenc} % Allow text to be searchable in output

% Other packages
\usepackage{appendix} % Appendix environment
\usepackage{nextpage} % Cleartooddpage command
%\usepackage[square,comma,sort,numbers]{natbib} % Reference formatting
\usepackage{setspace} % Line spacing
\usepackage{listings} % Display code with syntax highlighting
\usepackage{upquote} % Vertical quotes in verbatim
\usepackage{xcolor} % Colors
\usepackage{titlesec} % Header spacing
\usepackage{xparse} % for tcolorbox
\usepackage[listings]{tcolorbox} % Colored boxes for highlighting syntax
\tcbuselibrary{breakable}
\tcbuselibrary{skins}
\usepackage{enumitem} % better enumerate/itemize options
\usepackage{fancyhdr}
\usepackage{multicol}
\usepackage{ifthen}
\usepackage{xstring}

% Table of contents
\usepackage{imakeidx} % Index page
\usepackage{tocloft} % Control of table of contents
\usepackage[nottoc]{tocbibind} % Adds bibliography, table of tables, table of figures, to table of contents
\usepackage[bookmarks,linktocpage,hidelinks]{hyperref} % Hyperlinks for sections, figures, etc.

% Formatting
% Page format
\setlength{\oddsidemargin}{0.00in}  % Left side margin for odd numbered pages
\setlength{\evensidemargin}{0.00in} % Right side margin for even numbered pages
\setlength{\topmargin}{0.00in}      % Top margin
\setlength{\headheight}{0.20in}     % Header height
\setlength{\headsep}{0.20in}        % Separation between header and main text
\setlength{\topskip}{0.00in}        % Top skip
\setlength{\textwidth}{6.50in}      % Width of the text
\setlength{\textheight}{8.50in}     % Height of the text
\setlength{\footskip}{0.50in}       % Foot skip
\setlength{\parindent}{0.00in}      % First line indentation
\setlength{\parskip}{6pt}        % Space between two paragraphs

% Captions (figures, tables, etc.)
\setlength{\floatsep}{\parskip}          % Space left between floats.
\setlength{\textfloatsep}{\floatsep}   % Space between last top float
% or first bottom float and the text
\setlength{\intextsep}{\floatsep}      % Space left on top and bottom
% of an in-text float
\setlength{\abovecaptionskip}{0.1in plus 0.25in}  % Space above caption
\setlength{\belowcaptionskip}{0.1in plus 0.25in}  % Space below caption
\setlength{\captionmargin}{0.50in}     % Left/Right margin for caption
\setlength{\abovedisplayskip}{0.00in plus 0.25in} % Space before Math stuff
\setlength{\belowdisplayskip}{0.00in plus 0.25in} % Space after Math stuff
\setlength{\arraycolsep}{0.10in}       % Gap between columns of an array
\setlength{\jot}{0.10in}                % Gap between multiline equations
\setlength{\itemsep}{0.10in}           % Space between successive items

% Counters (no section numbering)
\setcounter{tocdepth}{3}
\setcounter{secnumdepth}{0}

% Spacing
\setstretch{1.5}

\titlespacing*{\section}{0cm}{6pt}{6pt}[0cm]
\titlespacing*{\subsection}{0cm}{6pt}{6pt}[0cm]
\titlespacing*{\subsubsection}{0cm}{6pt}{6pt}[0cm]

\titleformat{\section}
{\sffamily\huge}{}{0pt}{\titlerule\vspace{-0.2cm}}
\titleformat{\subsection}
{\sffamily\itshape\Large}{}{0pt}{}

% Macro for syntax
\newtcolorbox{syntax}{
    size=small,
    sharp corners,
    colframe=black,
    colback=yellow,
    fontupper=\bfseries\ttfamily
}

% Macro for argument table
\newenvironment{args}{
    \begin{tabular}{>{\bfseries\ttfamily}p{0.25\linewidth} p{0.69\linewidth}}
    }{
    \end{tabular}\par
    \vspace{0.5\baselineskip}
}

% Note: Requires packages "listing", "xcolor", and "textcomp"
\lstdefinelanguage{verbatim}{
    basicstyle=\ttfamily\small,
    xleftmargin=9pt,
    xrightmargin=9pt,
    columns=fullflexible,
    keepspaces=true,
    breaklines=true
}

% Example code
\AtBeginDocument{
\newtcolorbox[blend into=listings]{example}[2][]{
    colback=blue!3!white,
    colframe=black,
    colbacktitle=blue!15!white,
    coltitle=black,
    sharp corners,
    enhanced,
    breakable,
    size=small,
    before upper={
        \setstretch{1.0}\lstset{language=verbatim}\vspace{3pt}\textsf{\textit{Code:}}
    },
    subtitle style={
        colback=blue!20!white,
        fonttitle=\sffamily
    },
    before lower={
        \setstretch{1.0}\lstset{language=verbatim}\vspace{3pt}\textsf{\textit{Output:}}
    },
    fonttitle=\sffamily,
    title={#2},
    #1
}
}

% Links to sub and subsub commands - optional boolean argument, default true. if false, only displays subcmd.

% Commands (and command ensembles)
\newcommand{\command}[1]{\protect\hypertarget{#1}{#1}\index{#1}}
\newcommand{\subcommand}[2]{\protect\hypertarget{#1 #2}{#1 #2}\index{#1!#2}}
\newcommand{\cmdlink}[1]{\protect\hyperlink{#1}{\textit{#1}}}
\newcommand{\subcmdlink}[3][1]{\protect\hyperlink{#2 #3}{\ifnum#1=1\relax\textit{#2 #3}\else\textit{#3}\fi}}

% Methods (first arg is class)
\newcommand{\method}[2]{\protect\hypertarget{$#1Obj #2}{\$#1Obj #2}\index{#1 methods!#2}}
\newcommand{\methodlink}[3][1]{\protect\hyperlink{$#2Obj #3}{\ifnum#1=1\relax\textit{\$#2Obj #3}\else\textit{#3}\fi}}

% Macros for figure/table names
\newcommand{\fig}{\figurename\ }
\newcommand{\figs}{\figurename s }
\newcommand{\tbl}{\tablename\ }
\newcommand{\tbls}{\tablename s }
\newcommand{\eq}{Eq. }
\newcommand{\eqs}{Eqs. }
\renewcommand{\lstlistingname}{Example}% Listing -> Example
\renewcommand{\lstlistlistingname}{List of \lstlistingname s}% List of Listings -> List of Examples
\newcommand{\ex}{Example }
\newcommand{\exs}{Examples }
\newcommand{\var}[1]{\texttt{\textbf{\$#1}}}

% Header/footer
\renewcommand{\headrulewidth}{0pt}

% Changes to hyperlinks (URLs)
\renewcommand\UrlFont{\color{blue}\rmfamily}

% New column type 
% https://tex.stackexchange.com/questions/75717/how-can-i-mix-itemize-and-tabular-environments
\newcolumntype{L}{>{\labelitemi~~}l<{}}
\newcommand{\version}{0.1}

\renewcommand{\cleartooddpage}[1][]{\ignorespaces} % single side
\newcommand{\caret}{$^\wedge$}

\title{\Huge{Input/Output Utilities (iou)}\\\large Version \version}
\author{Alex Baker\\\small\url{https://github.com/ambaker1/iou}}
\date{\small\today}
\makeindex[columns=2,title={Command Index}]
\begin{document}
\maketitle
\begin{abstract}
\begin{center}
The package ``iou'' provides basic data import/export and conversion utilities.
Four datatypes are supported: space-delimited values (txt), comma-separated values (csv), matrices (mat), and tables (tbl). 
\end{center}
\end{abstract}
\clearpage
\section{File Input/Output}
The commands \cmdlink{fread} and \cmdlink{fputs} simplify file I/O in Tcl. 
\begin{syntax}
\command{fread} <\$option \$value ...> <-newline> \$file
\end{syntax}
\begin{args}
\$option \$value ... & File configuration options, see Tcl \textit{fconfigure} command. \\
-newline & Option to read the final newline if it exists. \\
\$file & File to read data from.
\end{args}
\begin{syntax}
\command{fputs} <\$option \$value ...> <-nonewline> \$file \$string
\end{syntax}
\begin{args}
\$option \$value ... & File configuration options, see Tcl \textit{fconfigure} command. \\
-nonewline & Option to not write a final newline. \\
\$file & File to write data to. \\
\$string & Data to write to file.
\end{args}
\begin{example}{File input/output}
\begin{lstlisting}
# Export data to file (creates or overwrites the file)
fputs example.txt "hello world"
# Import the contents of the file (requires that the file exists)
puts [fread example.txt]
\end{lstlisting}
\tcblower
\begin{lstlisting}
hello world
\end{lstlisting}
\end{example}

\clearpage
\section{Data Conversion}
This package also provides conversion utilities for different datatypes. 
The main datatype is matrix, or \textbf{mat}. 
\subsection{Matrix (mat)}
The matrix (\textbf{mat}) datatype is a nested Tcl list, where each list element represents a row vector of equal length.
This definition is compatible with the matrix data type provided by the \textcolor{blue}{\href{https://github.com/ambaker1/ndlist}{ndlist}} package. 

An example of a matrix with headers is shown below. 
\begin{example}{Example data (\textbf{mat}):}
\begin{lstlisting}
set mat {{step disp force} {1 0.02 4.5} {2 0.03 4.8} {3 0.07 12.6}}
\end{lstlisting}
\end{example}
This format can be converted from and to all other formats, as is illustrated in the diagram below, with ``\textbf{a}'' and ``\textbf{b}'' acting as placeholders for all other datatypes.
\begin{center}
\includegraphics{figures/dataconversion.pdf}
\end{center}
This way, each new datatype only requires the addition of two new conversion commands: one to \textbf{mat} and one from \textbf{mat}.
Then, you can convert between any datatype using \textbf{mat} as the intermediate datatype.
\clearpage
\subsection{Table (tbl)}
The table (\textbf{tbl}) datatype is a key-value paired list, with keys representing the table header, and values representing the columns.
This definition is compatible with the table data type provided by the \textcolor{blue}{\href{https://github.com/ambaker1/taboo}{taboo}} package.
To convert between \textbf{mat} and \textbf{tbl}, use the commands \cmdlink{mat2tbl} and \cmdlink{tbl2mat}.
\begin{syntax}
\command{mat2tbl} \$mat
\end{syntax}
\begin{syntax}
\command{tbl2mat} \$tbl
\end{syntax}
\begin{args}
\$mat & Matrix value. \\
\$tbl & Table value. 
\end{args}
\begin{example}{Example data (\textbf{tbl}):}
\begin{lstlisting}
puts [mat2tbl $mat]
\end{lstlisting}
\tcblower
\begin{lstlisting}
step {1 2 3} disp {0.02 0.03 0.07} force {4.5 4.8 12.6}
\end{lstlisting}
\end{example}
\clearpage
\subsection{Space-Delimited Text (txt)}
The space-delimited text (\textbf{txt}) datatype is simply space-delimited values, where new lines separate rows. 
Escaping of spaces and newlines is consistent with Tcl rules for valid lists. 
To convert between \textbf{mat} and \textbf{txt}, use the commands \cmdlink{mat2txt} and \cmdlink{txt2mat}. 
\begin{syntax}
\command{mat2txt} \$mat 
\end{syntax}
\begin{syntax}
\command{txt2mat} \$txt
\end{syntax}
\begin{args}
\$mat & Matrix value. \\
\$txt & Space-delimited values.
\end{args}
\begin{example}{Example data (\textbf{txt}):}
\begin{lstlisting}
puts [mat2txt $mat]
\end{lstlisting}
\tcblower
\begin{lstlisting}
step disp force
1 0.02 4.5
2 0.03 4.8
3 0.07 12.6
\end{lstlisting}
\end{example}
\clearpage
\subsection{Comma-Separated Values (csv)}
The comma-separated values (\textbf{csv}) datatype is comma delimited values, where new lines separate rows. 
Commas and newlines are escaped with quotes, and quotes are escaped with double-quotes. 
To convert between \textbf{mat} and \textbf{csv}, use the commands \cmdlink{mat2csv} and \cmdlink{csv2mat}. 
\begin{syntax}
\command{mat2csv} \$mat
\end{syntax}
\begin{syntax}
\command{csv2mat} \$csv
\end{syntax}
\begin{args}
\$mat & Matrix value. \\
\$csv & Comma-separated values.
\end{args}
\begin{example}{Example data (\textbf{csv}):}
\begin{lstlisting}
puts [mat2csv $mat]
\end{lstlisting}
\tcblower
\begin{lstlisting}
step,disp,force
1,0.02,4.5
2 0.03,4.8
3,0.07,12.6
\end{lstlisting}
\end{example}
\clearpage
\subsection{Derived Conversions}
Using the \textbf{mat} datatype as the intermediate datatype, data can be converted to and from any datatype. 
As a convenience, shortcuts are provided for conversions that use \textbf{mat} as an intermediate data format.
\begin{syntax}
\command{tbl2txt} \$tbl \\
\command{tbl2csv} \$tbl
\end{syntax}
\begin{syntax}
\command{txt2tbl} \$txt \\
\command{txt2csv} \$txt
\end{syntax}
\begin{syntax}
\command{csv2tbl} \$csv \\
\command{csv2txt} \$csv
\end{syntax}
\begin{args}
\$tbl & Table value. \\
\$txt & Space-delimited values. \\
\$csv & Comma-separated values.
\end{args}
\begin{example}{Combining data conversions}
\begin{lstlisting}
# Convert from table to csv, using mat as an intermediate datatype.
set tbl {step {1 2 3} disp {0.02 0.03 0.07} force {4.5 4.8 12.6}}
set csv [mat2csv [tbl2mat $tbl]]; # also could use tbl2csv
puts $csv
\end{lstlisting}
\tcblower
\begin{lstlisting}
step,disp,force
1,0.02,4.5
2,0.03,4.8
3,0.07,12.6
\end{lstlisting}
\end{example}
\clearpage
{\normalsize\printindex}
\end{document}





